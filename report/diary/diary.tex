\documentclass[12pt,letterpaper]{article}

\newcommand{\workingDate}{\textsc{2013 $|$ January $|$ 01}}
\newcommand{\userName}{Anthony S Chapman}
\newcommand{\institution}{University of Aberdeen}
\usepackage{researchdiary_png}
\usepackage{color}
% To add your university logo to the upper right, simply
% upload a file named "logo.png" using the files menu above.

\begin{document}
\univlogo


\textcolor{red}{\Huge Date}
\section*{section name} % (fold)
\label{sec:section_name}

% section section_name (end)

\textcolor{red}{\Huge 1st April 2015}
\section{Meeting: Steve + Pang} % (fold)
\label{sec:meeting}
During our meeting today, a future project was discussed: obesity prediction (regression) using AMND and STONES datasets. Hopefully we can find a computational way to solve this problem as supposed to purely statistical. 

Pang mention we could use machine (reinforced) learning for this. 
% section meeting (end)
\section{Clustering for predicting diseases} % (fold)
\label{sec:clustering_for_predicting_diseases}
The idea is to cluster the complete already imputed dataset and analyses the results. This should eliminate the bias that is created by specifically looking for an outcome (asthma for example). Once the data is split into different cluster, we can then analyse each cluster in terms on growth trajectories and how likely those trajectories will result in the different disorders. Anthony thought is to have give a likelihood of having any of the diseases in therms of the different trajectories (cluster x has trajectory y and they have likelihood of asthma 10\%, diabetes 20\% epilepsy 30\% etc..).
% section clustering_for_predicting_diseases (end)

\textcolor{blue}{\Huge 31 March 2015}
\section*{Latent Growth Modelling and Lavaan} % (fold)
\label{sec:latent_growth}
Given that we don't actually know the growth rate, we can tell a simple regression model to try and come up with a relationship between this and the measurements. Remember that regressions just look at the relationship between observed data. A latent variable is a variable that is unknown, thus we can specify that growth rate is the unknown variable and use this to model growth. 

Problem is that it is not easy for MICE to acknowledge Lavaan as a linear model when you pool the results.  
% section latent_growth (end)
\section*{Liner Mixed Effect (LME) Models} % (fold)
\label{sec:liner_mixed_effec_}
After speaking with one of my old Profs, I was suggested to use LMEs to find the growth trajectories. He also suggested that the LMEs could be used to find the missing values but I'm not so sure. We would like to use clustering for the detection of diseases, not statistical regressions. I believe they will be less bias this way as we won't specifically look for the solutions, we will look for any solution and arrive at the correct one. 

I will look into it to see if it can me used with MICE in the missing data stuff. 
% section liner_mixed_effec_ (end)
\section*{Git Hub} % (fold)
\label{sec:git_hub}
I have realised that I have not been paying attention to keeping a record and backups of everything I'm doing and have. Git Hub is a great way to back-up my work and also means I can access my documents from where ever I am. 
% section git_hub (end)
\section*{Latex} % (fold)
\label{sec:latex}
It has been about six months since I last used \LaTeX and I have become rusty! This diary will be a good way for me to stay in tune with academic writing. 
% section latex (end)
\section*{Diary} % (fold)
\label{sec:diary}
This diary will keep track of my process and also help me record papers and documents which I have read and thought important or worth remembering. 
% section diary (end)


\end{document}
